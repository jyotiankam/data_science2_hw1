\documentclass[]{article}
\usepackage{lmodern}
\usepackage{amssymb,amsmath}
\usepackage{ifxetex,ifluatex}
\usepackage{fixltx2e} % provides \textsubscript
\ifnum 0\ifxetex 1\fi\ifluatex 1\fi=0 % if pdftex
  \usepackage[T1]{fontenc}
  \usepackage[utf8]{inputenc}
\else % if luatex or xelatex
  \ifxetex
    \usepackage{mathspec}
  \else
    \usepackage{fontspec}
  \fi
  \defaultfontfeatures{Ligatures=TeX,Scale=MatchLowercase}
\fi
% use upquote if available, for straight quotes in verbatim environments
\IfFileExists{upquote.sty}{\usepackage{upquote}}{}
% use microtype if available
\IfFileExists{microtype.sty}{%
\usepackage{microtype}
\UseMicrotypeSet[protrusion]{basicmath} % disable protrusion for tt fonts
}{}
\usepackage[margin=1in]{geometry}
\usepackage{hyperref}
\hypersetup{unicode=true,
            pdftitle={Data Science II HW1},
            pdfauthor={Jyoti Ankam},
            pdfborder={0 0 0},
            breaklinks=true}
\urlstyle{same}  % don't use monospace font for urls
\usepackage{color}
\usepackage{fancyvrb}
\newcommand{\VerbBar}{|}
\newcommand{\VERB}{\Verb[commandchars=\\\{\}]}
\DefineVerbatimEnvironment{Highlighting}{Verbatim}{commandchars=\\\{\}}
% Add ',fontsize=\small' for more characters per line
\usepackage{framed}
\definecolor{shadecolor}{RGB}{248,248,248}
\newenvironment{Shaded}{\begin{snugshade}}{\end{snugshade}}
\newcommand{\KeywordTok}[1]{\textcolor[rgb]{0.13,0.29,0.53}{\textbf{#1}}}
\newcommand{\DataTypeTok}[1]{\textcolor[rgb]{0.13,0.29,0.53}{#1}}
\newcommand{\DecValTok}[1]{\textcolor[rgb]{0.00,0.00,0.81}{#1}}
\newcommand{\BaseNTok}[1]{\textcolor[rgb]{0.00,0.00,0.81}{#1}}
\newcommand{\FloatTok}[1]{\textcolor[rgb]{0.00,0.00,0.81}{#1}}
\newcommand{\ConstantTok}[1]{\textcolor[rgb]{0.00,0.00,0.00}{#1}}
\newcommand{\CharTok}[1]{\textcolor[rgb]{0.31,0.60,0.02}{#1}}
\newcommand{\SpecialCharTok}[1]{\textcolor[rgb]{0.00,0.00,0.00}{#1}}
\newcommand{\StringTok}[1]{\textcolor[rgb]{0.31,0.60,0.02}{#1}}
\newcommand{\VerbatimStringTok}[1]{\textcolor[rgb]{0.31,0.60,0.02}{#1}}
\newcommand{\SpecialStringTok}[1]{\textcolor[rgb]{0.31,0.60,0.02}{#1}}
\newcommand{\ImportTok}[1]{#1}
\newcommand{\CommentTok}[1]{\textcolor[rgb]{0.56,0.35,0.01}{\textit{#1}}}
\newcommand{\DocumentationTok}[1]{\textcolor[rgb]{0.56,0.35,0.01}{\textbf{\textit{#1}}}}
\newcommand{\AnnotationTok}[1]{\textcolor[rgb]{0.56,0.35,0.01}{\textbf{\textit{#1}}}}
\newcommand{\CommentVarTok}[1]{\textcolor[rgb]{0.56,0.35,0.01}{\textbf{\textit{#1}}}}
\newcommand{\OtherTok}[1]{\textcolor[rgb]{0.56,0.35,0.01}{#1}}
\newcommand{\FunctionTok}[1]{\textcolor[rgb]{0.00,0.00,0.00}{#1}}
\newcommand{\VariableTok}[1]{\textcolor[rgb]{0.00,0.00,0.00}{#1}}
\newcommand{\ControlFlowTok}[1]{\textcolor[rgb]{0.13,0.29,0.53}{\textbf{#1}}}
\newcommand{\OperatorTok}[1]{\textcolor[rgb]{0.81,0.36,0.00}{\textbf{#1}}}
\newcommand{\BuiltInTok}[1]{#1}
\newcommand{\ExtensionTok}[1]{#1}
\newcommand{\PreprocessorTok}[1]{\textcolor[rgb]{0.56,0.35,0.01}{\textit{#1}}}
\newcommand{\AttributeTok}[1]{\textcolor[rgb]{0.77,0.63,0.00}{#1}}
\newcommand{\RegionMarkerTok}[1]{#1}
\newcommand{\InformationTok}[1]{\textcolor[rgb]{0.56,0.35,0.01}{\textbf{\textit{#1}}}}
\newcommand{\WarningTok}[1]{\textcolor[rgb]{0.56,0.35,0.01}{\textbf{\textit{#1}}}}
\newcommand{\AlertTok}[1]{\textcolor[rgb]{0.94,0.16,0.16}{#1}}
\newcommand{\ErrorTok}[1]{\textcolor[rgb]{0.64,0.00,0.00}{\textbf{#1}}}
\newcommand{\NormalTok}[1]{#1}
\usepackage{graphicx,grffile}
\makeatletter
\def\maxwidth{\ifdim\Gin@nat@width>\linewidth\linewidth\else\Gin@nat@width\fi}
\def\maxheight{\ifdim\Gin@nat@height>\textheight\textheight\else\Gin@nat@height\fi}
\makeatother
% Scale images if necessary, so that they will not overflow the page
% margins by default, and it is still possible to overwrite the defaults
% using explicit options in \includegraphics[width, height, ...]{}
\setkeys{Gin}{width=\maxwidth,height=\maxheight,keepaspectratio}
\IfFileExists{parskip.sty}{%
\usepackage{parskip}
}{% else
\setlength{\parindent}{0pt}
\setlength{\parskip}{6pt plus 2pt minus 1pt}
}
\setlength{\emergencystretch}{3em}  % prevent overfull lines
\providecommand{\tightlist}{%
  \setlength{\itemsep}{0pt}\setlength{\parskip}{0pt}}
\setcounter{secnumdepth}{0}
% Redefines (sub)paragraphs to behave more like sections
\ifx\paragraph\undefined\else
\let\oldparagraph\paragraph
\renewcommand{\paragraph}[1]{\oldparagraph{#1}\mbox{}}
\fi
\ifx\subparagraph\undefined\else
\let\oldsubparagraph\subparagraph
\renewcommand{\subparagraph}[1]{\oldsubparagraph{#1}\mbox{}}
\fi

%%% Use protect on footnotes to avoid problems with footnotes in titles
\let\rmarkdownfootnote\footnote%
\def\footnote{\protect\rmarkdownfootnote}

%%% Change title format to be more compact
\usepackage{titling}

% Create subtitle command for use in maketitle
\newcommand{\subtitle}[1]{
  \posttitle{
    \begin{center}\large#1\end{center}
    }
}

\setlength{\droptitle}{-2em}

  \title{Data Science II HW1}
    \pretitle{\vspace{\droptitle}\centering\huge}
  \posttitle{\par}
    \author{Jyoti Ankam}
    \preauthor{\centering\large\emph}
  \postauthor{\par}
    \date{}
    \predate{}\postdate{}
  

\begin{document}
\maketitle

\begin{Shaded}
\begin{Highlighting}[]
\KeywordTok{library}\NormalTok{(tidyverse)}
\KeywordTok{library}\NormalTok{(glmnet)}
\KeywordTok{library}\NormalTok{(caret)}
\KeywordTok{library}\NormalTok{(ISLR)}
\KeywordTok{library}\NormalTok{(corrplot)}
\KeywordTok{library}\NormalTok{(plotmo)}
\KeywordTok{library}\NormalTok{(pls)}
\end{Highlighting}
\end{Shaded}

Reading/loading the datasets:

\begin{Shaded}
\begin{Highlighting}[]
\NormalTok{train_solubility <-}\StringTok{ }\KeywordTok{read.csv}\NormalTok{(}\StringTok{"C:/Users/19293/Desktop/Data Science II/solubility_train.csv"}\NormalTok{) }

\NormalTok{test_solubility <-}\StringTok{ }\KeywordTok{read.csv}\NormalTok{(}\StringTok{"C:/Users/19293/Desktop/Data Science II/solubility_test.csv"}\NormalTok{)}
\end{Highlighting}
\end{Shaded}

The data has been divided into training and test datasets containing 229
variables each. There are 951 observations in the training dataset and
316 observations in the test dataset.

\section{Least sqaures}\label{least-sqaures}

Fitting the linear model using least sqaures on the training data:

\begin{Shaded}
\begin{Highlighting}[]
\NormalTok{fit1 <-}\StringTok{ }\KeywordTok{lm}\NormalTok{(Solubility }\OperatorTok{~}\StringTok{ }\NormalTok{. , }\DataTypeTok{data =}\NormalTok{ train_solubility)}

\NormalTok{fit2 <-}\StringTok{ }\KeywordTok{lm}\NormalTok{(Solubility }\OperatorTok{~}\StringTok{ }\NormalTok{. , }\DataTypeTok{data =}\NormalTok{ test_solubility)}

\KeywordTok{mean}\NormalTok{((test_solubility}\OperatorTok{$}\NormalTok{Solubility }\OperatorTok{-}\StringTok{ }\KeywordTok{predict.lm}\NormalTok{(fit1, test_solubility)) }\OperatorTok{^}\StringTok{ }\DecValTok{2}\NormalTok{)}
\end{Highlighting}
\end{Shaded}

\begin{verbatim}
## [1] 0.5558898
\end{verbatim}

The mean square error using the test data is 0.5558

\section{Ridge regression}\label{ridge-regression}

Fitting a ridge regression model on the training data, with λ chosen by
cross-validation.

\begin{Shaded}
\begin{Highlighting}[]
\NormalTok{X.Training =}\StringTok{ }\KeywordTok{model.matrix}\NormalTok{(Solubility }\OperatorTok{~}\StringTok{ }\NormalTok{., train_solubility)[,}\OperatorTok{-}\DecValTok{1}\NormalTok{]}
\NormalTok{Y.Training =}\StringTok{ }\NormalTok{train_solubility}\OperatorTok{$}\NormalTok{Solubility}

\NormalTok{ridge.mod <-}\StringTok{ }\KeywordTok{glmnet}\NormalTok{(X.Training, Y.Training, }\DataTypeTok{alpha=}\DecValTok{0}\NormalTok{, }\DataTypeTok{lambda =} \KeywordTok{exp}\NormalTok{(}\KeywordTok{seq}\NormalTok{(}\OperatorTok{-}\DecValTok{10}\NormalTok{, }\DecValTok{10}\NormalTok{, }\DataTypeTok{length=}\DecValTok{200}\NormalTok{)))}
\KeywordTok{plot}\NormalTok{(ridge.mod)}
\end{Highlighting}
\end{Shaded}

\includegraphics{jva2106_hw1_files/figure-latex/unnamed-chunk-4-1.pdf}

\begin{Shaded}
\begin{Highlighting}[]
\KeywordTok{set.seed}\NormalTok{(}\DecValTok{2}\NormalTok{)}
\NormalTok{cv.ridge <-}\StringTok{ }\KeywordTok{cv.glmnet}\NormalTok{(X.Training, Y.Training, }
                      \DataTypeTok{alpha =} \DecValTok{0}\NormalTok{, }
                      \DataTypeTok{lambda =} \KeywordTok{exp}\NormalTok{(}\KeywordTok{seq}\NormalTok{(}\OperatorTok{-}\DecValTok{10}\NormalTok{, }\DecValTok{10}\NormalTok{, }\DataTypeTok{length=}\DecValTok{200}\NormalTok{)),}
                      \DataTypeTok{type.measure =} \StringTok{"mse"}\NormalTok{)}

\KeywordTok{plot}\NormalTok{(cv.ridge)}
\end{Highlighting}
\end{Shaded}

\includegraphics{jva2106_hw1_files/figure-latex/unnamed-chunk-4-2.pdf}

\begin{Shaded}
\begin{Highlighting}[]
\NormalTok{optimal_lamda =}\StringTok{ }\NormalTok{cv.ridge}\OperatorTok{$}\NormalTok{lambda.min}
\end{Highlighting}
\end{Shaded}

Using the lambda chosen by cross validation:

\begin{Shaded}
\begin{Highlighting}[]
\NormalTok{X.Testing =}\StringTok{ }\KeywordTok{model.matrix}\NormalTok{(Solubility }\OperatorTok{~}\StringTok{ }\NormalTok{., test_solubility)[,}\OperatorTok{-}\DecValTok{1}\NormalTok{]}
\NormalTok{Y.Testing =}\StringTok{ }\NormalTok{test_solubility}\OperatorTok{$}\NormalTok{Solubility}

\KeywordTok{set.seed}\NormalTok{(}\DecValTok{2}\NormalTok{)}
\NormalTok{pred =}\StringTok{ }\KeywordTok{predict}\NormalTok{(ridge.mod, }\DataTypeTok{s =}\NormalTok{ optimal_lamda, }\DataTypeTok{newx =}\NormalTok{ X.Testing)}
\KeywordTok{mean}\NormalTok{((pred }\OperatorTok{-}\StringTok{ }\NormalTok{Y.Testing)}\OperatorTok{^}\DecValTok{2}\NormalTok{)}
\end{Highlighting}
\end{Shaded}

\begin{verbatim}
## [1] 0.5115182
\end{verbatim}

The mean test error is 0.5115

\section{Lasso}\label{lasso}

Fitting a lasso model on the training data, with λ chosen by
cross-validation

\begin{Shaded}
\begin{Highlighting}[]
\NormalTok{X.Train =}\StringTok{ }\KeywordTok{model.matrix}\NormalTok{(Solubility }\OperatorTok{~}\StringTok{ }\NormalTok{., train_solubility)[,}\OperatorTok{-}\DecValTok{1}\NormalTok{]}
\NormalTok{Y.Train =}\StringTok{ }\NormalTok{train_solubility}\OperatorTok{$}\NormalTok{Solubility}

\NormalTok{lasso.mod <-}\StringTok{ }\KeywordTok{glmnet}\NormalTok{(X.Train, Y.Train, }\DataTypeTok{alpha=}\DecValTok{1}\NormalTok{, }\DataTypeTok{lambda =} \KeywordTok{exp}\NormalTok{(}\KeywordTok{seq}\NormalTok{(}\OperatorTok{-}\DecValTok{10}\NormalTok{, }\DecValTok{10}\NormalTok{, }\DataTypeTok{length=}\DecValTok{200}\NormalTok{)))}
\KeywordTok{plot}\NormalTok{(lasso.mod)}
\end{Highlighting}
\end{Shaded}

\includegraphics{jva2106_hw1_files/figure-latex/unnamed-chunk-6-1.pdf}

\begin{Shaded}
\begin{Highlighting}[]
\KeywordTok{set.seed}\NormalTok{(}\DecValTok{2}\NormalTok{)}
\NormalTok{cv.lasso <-}\StringTok{ }\KeywordTok{cv.glmnet}\NormalTok{(X.Train, Y.Train, }
                      \DataTypeTok{alpha =} \DecValTok{1}\NormalTok{, }
                      \DataTypeTok{lambda =} \KeywordTok{exp}\NormalTok{(}\KeywordTok{seq}\NormalTok{(}\OperatorTok{-}\DecValTok{10}\NormalTok{, }\DecValTok{10}\NormalTok{, }\DataTypeTok{length=}\DecValTok{200}\NormalTok{)),}
                      \DataTypeTok{type.measure =} \StringTok{"mse"}\NormalTok{)}

\KeywordTok{plot}\NormalTok{(cv.ridge)}
\end{Highlighting}
\end{Shaded}

\includegraphics{jva2106_hw1_files/figure-latex/unnamed-chunk-6-2.pdf}

\begin{Shaded}
\begin{Highlighting}[]
\NormalTok{optimum_lamda =}\StringTok{ }\NormalTok{cv.lasso}\OperatorTok{$}\NormalTok{lambda.min}
\end{Highlighting}
\end{Shaded}

\begin{Shaded}
\begin{Highlighting}[]
\NormalTok{X.Test =}\StringTok{ }\KeywordTok{model.matrix}\NormalTok{(Solubility }\OperatorTok{~}\StringTok{ }\NormalTok{., test_solubility)[,}\OperatorTok{-}\DecValTok{1}\NormalTok{]}
\NormalTok{Y.Test =}\StringTok{ }\NormalTok{test_solubility}\OperatorTok{$}\NormalTok{Solubility}

\KeywordTok{set.seed}\NormalTok{(}\DecValTok{2}\NormalTok{)}
\NormalTok{pred2 =}\StringTok{ }\KeywordTok{predict}\NormalTok{(lasso.mod, }\DataTypeTok{s =}\NormalTok{ optimum_lamda, }\DataTypeTok{newx =}\NormalTok{ X.Test)}
\KeywordTok{mean}\NormalTok{((pred2 }\OperatorTok{-}\StringTok{ }\NormalTok{Y.Test)}\OperatorTok{^}\DecValTok{2}\NormalTok{)}
\end{Highlighting}
\end{Shaded}

\begin{verbatim}
## [1] 0.4988678
\end{verbatim}

The test error is 0.4988.

\begin{Shaded}
\begin{Highlighting}[]
\NormalTok{lasso_coef =}\StringTok{ }\KeywordTok{predict}\NormalTok{(cv.lasso, }\DataTypeTok{s=}\StringTok{"lambda.min"}\NormalTok{, }\DataTypeTok{type=}\StringTok{"coefficients"}\NormalTok{)}
\KeywordTok{length}\NormalTok{(lasso_coef[lasso_coef }\OperatorTok{!=}\StringTok{ }\DecValTok{0}\NormalTok{])}
\end{Highlighting}
\end{Shaded}

\begin{verbatim}
## [1] 144
\end{verbatim}

There are 144 non-zero coefficient estimates

\subsubsection{PCR}\label{pcr}

Fitting a PCR model on the training data, with M chosen by
cross-validation.

\begin{Shaded}
\begin{Highlighting}[]
\NormalTok{ctrl1 <-}\StringTok{ }\KeywordTok{trainControl}\NormalTok{(}\DataTypeTok{method =} \StringTok{"repeatedcv"}\NormalTok{, }\DataTypeTok{number =} \DecValTok{10}\NormalTok{, }\DataTypeTok{repeats =} \DecValTok{5}\NormalTok{)}
\KeywordTok{set.seed}\NormalTok{(}\DecValTok{2}\NormalTok{)}
\NormalTok{pcr.fit <-}\StringTok{ }\KeywordTok{train}\NormalTok{(X.Training,Y.Training,}
                 \DataTypeTok{method =} \StringTok{"pcr"}\NormalTok{,}
                 \DataTypeTok{tuneLength =} \DecValTok{228}\NormalTok{,}
                 \DataTypeTok{trControl =}\NormalTok{ ctrl1,}
                 \DataTypeTok{scale =} \OtherTok{TRUE}\NormalTok{)}

\NormalTok{predy2.pcr2 <-}\StringTok{ }\KeywordTok{predict}\NormalTok{(pcr.fit}\OperatorTok{$}\NormalTok{finalModel, }\DataTypeTok{newdata =}\NormalTok{ X.Testing, }
                       \DataTypeTok{ncomp =}\NormalTok{ pcr.fit}\OperatorTok{$}\NormalTok{bestTune}\OperatorTok{$}\NormalTok{ncomp)}
\KeywordTok{mean}\NormalTok{((predy2.pcr2}\OperatorTok{-}\NormalTok{Y.Testing)}\OperatorTok{^}\DecValTok{2}\NormalTok{)}
\end{Highlighting}
\end{Shaded}

\begin{verbatim}
## [1] 0.540555
\end{verbatim}

\begin{Shaded}
\begin{Highlighting}[]
\KeywordTok{ggplot}\NormalTok{(pcr.fit, }\DataTypeTok{highlight =} \OtherTok{TRUE}\NormalTok{) }\OperatorTok{+}\StringTok{ }\KeywordTok{theme_bw}\NormalTok{()}
\end{Highlighting}
\end{Shaded}

\includegraphics{jva2106_hw1_files/figure-latex/unnamed-chunk-9-1.pdf}
The test error is 0.5405 and is lowest at M = 149

\section{Briefly discuss the results obtained in
(a)∼(d)}\label{briey-discuss-the-results-obtained-in-ad}

From the models above, the RMSE is least for the lasso model compared to
ridge regression, least squares linear method and PCR. It means that the
coefficient for some of the predictors are truly zero. Additionally,
lasso is a more restrictive model, and thus it has the possibility of
reducing overfitting and variance in predictions. As long as it does not
result in too high of a bias due to its added constraints, it will
outperform least squares and the other methods which might be fitting
spurious parameters.


\end{document}
